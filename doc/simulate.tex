\documentclass{article}

\title{Fisher's method for protein inference}

%\title{Present or correct, what is the difference between different
% lists of proteins?}

%\title{Which questions do you answer with a list of proteins?}

\setlength{\parindent}{0pt}

\author{Matthew The}

\usepackage{algorithm}
\usepackage[noend]{algpseudocode}
\usepackage{amsmath}
\usepackage{url}
\usepackage{bbm}
\usepackage{graphicx}
\usepackage[outdir=./img/]{epstopdf}
\usepackage{epsfig}
\usepackage{a4wide}

\begin{document}

\maketitle

\section{Finding proteins with equal or subset peptide sets}

Input: 
\begin{itemize}
 \item fasta file with protein sequences
\end{itemize}

Output:
\begin{itemize}
 \item a map $M_F: \mbox{fragment protein} \to \mbox{representing
protein}$
 \item a map $M_D: \mbox{duplicate protein} \to \mbox{representing
protein}$
\end{itemize}

\bigskip
Algorithm:
\begin{enumerate}
 \item Create a map $M_P: \mbox{peptide} \to \mbox{list of proteins
peptide occurs in}$.
 \item Digest proteins with appropriate parameters (minimum peptide
length, maximum peptide length, maximum miscleavages, protease) and
fill $M_P$ with the results.
 \item Create a map $M_S: \mbox{protein} \to \mbox{set of proteins
whose peptides are a subset of the protein}$.
 \item Create a map $M_L: \mbox{protein} \to \mbox{number of peptides
in the protein}$.
 \item For each protein $R$:
 \begin{enumerate}
  \item Extract the protein lists from $M_P$ for all peptides in $R$
and take the intersection. This intersection $I$ contains all
proteins for which $R$'s peptide set is a subset (could be equal as
well).
  \item Remove $R$ from $I$
  \item If no proteins are left in $I$ go to the next
protein.
  \item Sort the proteins in $I$ in alphabetical order. Let $Z$ be the
last protein in this list.
  \item Add the set of proteins under the key $R$ in $M_S$ to the set
of proteins under the key $Z$ in $M_S$. Remove the key $R$ from $M_S$.
  \item Add $R$ to the set of proteins at key $Z$ in $M_S$.
  \item Add an entry with the number of peptides for $R$ in $M_L$.
 \end{enumerate}
 \item For each protein key $K$ in $M_S$:
 \begin{enumerate}
  \item For each protein $C$ in the set of proteins belonging to $K$:
  \begin{enumerate}
   \item if $C$ has the same number of peptides as $K$ (use $M_L$ to
check), add an entry in $M_D$ with key $C$ and value $K$.
   \item if $C$ has fewer peptides than $K$, add an entry in $M_F$
with key $C$ and value $K$.
  \end{enumerate}
 \end{enumerate}
\end{enumerate}

\section{Removing shared peptides from percolator peptide list}

Input: 
\begin{itemize}
 \item list of peptides $L_A$, each with a list of proteins it
appears in.
 \item a map $M_F: \mbox{fragment protein} \to \mbox{representing
protein}$
 \item a map $M_D: \mbox{duplicate protein} \to \mbox{representing
protein}$
\end{itemize}

Output:
\begin{itemize}
 \item list of non-shared peptides $L_N$, each with a list of proteins
it appears in.
\end{itemize}

\bigskip
Algorithm:
\begin{enumerate}
 \item For each peptide $P$ in $L_A$:
 \begin{enumerate}
  \item Create a set $S_R$ of representing proteins
  \item For each protein $C$ in the protein list of $P$:
  \begin{enumerate}
   \item Use $M_D$ and $M_L$ to find the representing protein $R$ of
$C$.
   \item Add $R$ to $S_R$.
  \end{enumerate}
  \item If $S_R$ contains just $1$ protein, $P$ is non-shared
and can be added to $L_N$, otherwise we throw it out.
 \end{enumerate}
\end{enumerate}
\end{document}